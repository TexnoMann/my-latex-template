% Основной файл, содержащий форматирование.
% Для начинающих:
% Синтаксис LaTeX:
% https://en.wikibooks.org/wiki/LaTeX/Basics#The_LaTeX_syntax;
% Почему для сборки используем xelatex? Потому что на дворе 2020 год,
% а xelatex все еще один из немногих, кто предоставляет поддержку текста в
% юникоде.
% https://ru.wikibooks.org/wiki/LaTeX/Форматирование_текста - а вот здесь
% описано, как делать кастомные переносы.

% Выбираем тип документа. Я использую extreport по рекомендации Столярова.
% Подробнее об опциях documentclass:
% https://texblog.org/2013/02/13/latex-documentclass-options-illustrated/
% опцию final можно заменять на draft, чтобы собрать док в драфт-моде.
% можно использовать notitlepage, если не нужен титульник.
% Возможнозаменить на extarticle, если не используются главы.
\documentclass[14pt,a4paper,final,oneside]{extreport}

% -------- ЯЗЫКИ и ШРИФТЫ --------

% Подгружаем xltxtra. Подробнее о том, что он дает:
% https://ru.wikibooks.org/wiki/LaTeX/XeLaTeX
\usepackage{xltxtra}
% включаем лигатуры обычного теха (например '--' создаст нам короткое тире)
\defaultfontfeatures{Ligatures=TeX}

% xltxtra позволит нам использовать нормальные шрифты (TrueType, OpenType). Как
% итог, документ, набранный в utf-8, будет смотреться на ура. ОБЯЗАТЕЛЬНО
% проверьте, что ваши исходники в utf-8. Линукс юзерам советую проверить,
% установлены ли Times, Arial и Courier у вас в системе.
\setmainfont{Times New Roman}
\setromanfont{Times New Roman}
\setsansfont{Arial}
\setmonofont{Courier New}

% А теперь давайте включим нормальные переносы и еще пару мелких фич:
% https://ru.wikibooks.org/wiki/LaTeX/Использование_разных_языков
% если переносы вам не нужны, не комментируйте эту строку а используйте
% \usepackage[english,russian]{babel}
% https://ru.wikibooks.org/wiki/LaTeX/polyglossia
\usepackage{fontspec}
\usepackage{csquotes}
\usepackage{polyglossia}
\setmainlanguage{russian}
\setotherlanguage{english}
\setkeys{russian}{babelshorthands=true}
\newfontfamily{\cyrillicfont}{Times New Roman}
\newfontfamily{\cyrillicfontrm}{Times New Roman}
\newfontfamily{\cyrillicfonttt}{Courier New}
\newfontfamily{\cyrillicfontsf}{Arial}
% Заодно автоматом установим подписи к различным элементам (рисункам, таблицам).
\addto\captionsrussian{%
  \renewcommand{\figurename}{Рисунок}%
  \renewcommand{\contentsname}{\textnormal{Содержание}}%
}

% -------- ГЕОМЕТРИЯ И ОТСТУПЫ --------

% Полуторный интервал сделаем через setspace. Подробнее о том, как сжимать
% определенные части текста (например, текст в титульнике) с другим интервалом:
% https://ru.wikibooks.org/wiki/LaTeX/Форматирование_текста#Межстрочный_интервал
\usepackage{setspace}
\onehalfspacing

% Сделаем нужные отступы. Выдержка из ГОСТ 7.32-2017:
% Текст отчета следует печатать, соблюдая следующие размеры полей:
% левое - 30 мм,
% правое - 15 мм,
% верхнее и нижнее - 20 мм.
\usepackage[left=3cm,right=1.5cm,top=2cm,bottom=2cm]{geometry}
% Абзацный отступ должен быть
% одинаковым по всему тексту отчета и равен 1,25 см.
\setlength{\parindent}{1.25cm}

% Здесь человек предлагает использовать indentfirst для авто-отступа
% в параграфе: http://mydebianblog.blogspot.com/2006/11/latex-usepackage.html
\usepackage{indentfirst}

% Перечисления. За примером смотрим пункт 6.4.6.
\usepackage{enumitem}
\usepackage{calc}% используется для сложения длин
\setlist[itemize]{%
    leftmargin=0pt, % согласно ГОСТ, вторая строка элемента списка должна
% начинаться без абзацного отступа.
%
% Отступ первой строки от левого края будет равен: абз. отступ + ' -- '.
% Хочу заметить, что это результат не совсем точный, но выглядит неплохо.
% (другого способа настроить нужный отступ не нашел. ¯\_(ツ)_/¯). TODO.
    itemindent={\the\parindent + \widthof{\ --\ }},
    itemsep=0cm, % лол, не знаю что это;
    nolistsep, % убираем большие скачки по вертикали;
    label=--% используем короткое тире вместо bullets.
}
\setlist[enumerate]{%
% Итоговый отступ элемента от левого края будет: 1.5cm + ширина ' 1) '.
% В результате на элементе из двух цифр, типа '10)', может вылезти за края 😨.
    leftmargin=0pt,
    itemindent={\the\parindent + \widthof{\ 1)\ }},
    itemsep=0cm,
    nolistsep,
    label={\arabic*)}%
}

% -------- ФОРМАТИРОВАНИЕ ЗАГОЛОВКОВ РАЗДЕЛОВ --------

% Пункт 6.2.4: Переносы слов в заголовках не допускаются. Отключаем их.
\usepackage[raggedright,explicit]{titlesec}
% Форматируем заголовки элементов. См. подробности в документации titlesec
\titleformat{\chapter}[block]
{\bfseries}% жирный шрифт
{\hspace{1.25cm}\thechapter}% label
{\widthof{\ \ }}% два пробела между номером и названием раздела
{#1}% before-code

% Специальный случай для структурных элементов См. 6.2.1
\titleformat{name=\chapter,numberless}[block]
{\bfseries\centering}%
{}%
{0pt}%
{\MakeUppercase{#1}}

\titleformat{\section}[block]
{\bfseries}% жирный шрифт
{\hspace{1.25cm}\thesection}% label
{\widthof{\ \ }}% два пробела между номером и названием подраздела
{#1}% before-code

% На всякий пожарный, numberless-версия
\titleformat{name=\section,numberless}[block]
{\bfseries}% жирный шрифт
{\hspace{1.25cm}}% label
{\widthof{\ \ }}% два пробела между номером и названием подраздела
{#1}% before-code

% С пунктами сложнее всего. Согласно гост, они ведь и заголовки иметь могут...
% Не рекомендую, в общем, их использовать.
\titleformat{\subsection}[runin]
{\normalfont}% обычный шрифт
{\hspace{1.25cm}\thesubsection}% label
{\widthof{\ \ }}% два пробела между номером и названием подраздела
{#1}
% {\ifblank{#1}{}{%
%     \addcontentsline{toc}{subsection}{#1}
%     #1 \\ \indent%
% }}% before-code

% Теперь нормально расположим элементы по горизонтали
% \titlespacing{command}{left}{before-sep}{after-sep}
\titlespacing{\chapter}{0pt}{-20pt}{18pt}
\titlespacing{\section}{0pt}{18pt}{0pt}
\titlespacing{\subsection}{0pt}{0pt}{\wordsep}

% -------- ФОРМАТИРОВАНИЕ СОДЕРЖАНИЯ --------

% Отступы
\makeatletter
\renewcommand{\l@chapter}{\@dottedtocline{1}{2ex}{2ex}}
\renewcommand{\l@section}{\@dottedtocline{1}{4ex}{4ex}}
% \renewcommand{\l@subsection}{\@dottedtocline{1}{6ex}{6ex}}
\makeatother

% Не отображаем пункты. Что уж тут поделать.
\setcounter{tocdepth}{1}

% -------- ИЛЛЮСТРАЦИИ И ТАБЛИЦЫ --------

% предоставляет нам возможность вставлять картинки через \includegraphics
% https://www.overleaf.com/learn/latex/Inserting_Images
\usepackage{graphicx}
% Объявляем поддерживаемые форматы
\DeclareGraphicsExtensions{.pdf,.png,.jpg}

% Настройка подписей: убираем пустую строку, ставим дефис как разделитель
\usepackage[labelsep=endash]{caption}
% Смещаем подпись к таблице влево
\captionsetup[table]{singlelinecheck=false,justification=raggedright}
% Убираем пустую строку после подписей к рисункам
% \captionsetup[figure]{belowskip=-14pt,aboveskip=0pt}

% Используем расширенные таблицы (tabularx + longtable в одном).
% Вот здесь замечательный ответ, решающий большинство проблем по ним.
% https://tex.stackexchange.com/questions/59309/ltablex-customize-caption
\usepackage{ltablex}

% -------- БИБЛИОГРАФИЯ --------

% \usepackage{csquotes}
% Подключаем Библатех
\usepackage[
    backend=biber,
    sorting=none,
    style=gost-numeric
]{biblatex}

% Переопределяем заголовок списка источников
\DefineBibliographyStrings{russian}{%
  bibliography=\textnormal{Список использованных источников},
}

% Сбрасываем особенный стиль ссылок
\usepackage{url}
\urlstyle{same}

% -------- МИКРО-УЛУЧШЕНИЯ --------

% Избегает "линий-одиночек" (но порой создает пустые строки в конце страницы).
\usepackage[all]{nowidow}
% ну, или так:
% \widowpenalty=10000
% \clubpenalty=10000

% Интересный пакет. Чтобы понять его прелесть, соберите два документа -
% с этим пакетом и без, откройте их side-by-side, и внимательно посмотрите,
% чем они отличаются.
\usepackage{microtype}

% Гиперссылки
% hidelinks скрывает бесячую (неотключаемую?) подсветку ссылок в Acrobat
% и Firefox.
\usepackage[pdfusetitle,hidelinks]{hyperref}

% Условные выражения в коде (см. \newapp)
\usepackage{etoolbox}

% Позволяет использовать нижнее подчеркивание ("_") без слэша.
\usepackage{underscore}
