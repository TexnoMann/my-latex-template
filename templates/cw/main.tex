% Все отсюда до \begin{document} называется преамбулой.
% Подключаем файл ГОСТ-стиля
% Основной файл, содержащий форматирование.
% Для начинающих:
% Синтаксис LaTeX:
% https://en.wikibooks.org/wiki/LaTeX/Basics#The_LaTeX_syntax;
% Почему для сборки используем xelatex? Потому что на дворе 2020 год,
% а xelatex все еще один из немногих, кто предоставляет поддержку текста в
% юникоде.
% https://ru.wikibooks.org/wiki/LaTeX/Форматирование_текста - а вот здесь
% описано, как делать кастомные переносы.

% Выбираем тип документа. Я использую extreport по рекомендации Столярова.
% Подробнее об опциях documentclass:
% https://texblog.org/2013/02/13/latex-documentclass-options-illustrated/
% опцию final можно заменять на draft, чтобы собрать док в драфт-моде.
% можно использовать notitlepage, если не нужен титульник.
% Возможнозаменить на extarticle, если не используются главы.
\documentclass[14pt,a4paper,final,oneside]{extreport}

% -------- ЯЗЫКИ и ШРИФТЫ --------

% Подгружаем xltxtra. Подробнее о том, что он дает:
% https://ru.wikibooks.org/wiki/LaTeX/XeLaTeX
\usepackage{xltxtra}
% включаем лигатуры обычного теха (например '--' создаст нам короткое тире)
\defaultfontfeatures{Ligatures=TeX}

% xltxtra позволит нам использовать нормальные шрифты (TrueType, OpenType). Как
% итог, документ, набранный в utf-8, будет смотреться на ура. ОБЯЗАТЕЛЬНО
% проверьте, что ваши исходники в utf-8. Линукс юзерам советую проверить,
% установлены ли Times, Arial и Courier у вас в системе.
\setmainfont{Times New Roman}
\setromanfont{Times New Roman}
\setsansfont{Arial}
\setmonofont{Courier New}

% А теперь давайте включим нормальные переносы и еще пару мелких фич:
% https://ru.wikibooks.org/wiki/LaTeX/Использование_разных_языков
% если переносы вам не нужны, не комментируйте эту строку а используйте
% \usepackage[english,russian]{babel}
% https://ru.wikibooks.org/wiki/LaTeX/polyglossia
\usepackage{fontspec}
\usepackage{csquotes}
\usepackage{polyglossia}
\setmainlanguage{russian}
\setotherlanguage{english}
\setkeys{russian}{babelshorthands=true}
\newfontfamily{\cyrillicfont}{Times New Roman}
\newfontfamily{\cyrillicfontrm}{Times New Roman}
\newfontfamily{\cyrillicfonttt}{Courier New}
\newfontfamily{\cyrillicfontsf}{Arial}
% Заодно автоматом установим подписи к различным элементам (рисункам, таблицам).
\addto\captionsrussian{%
  \renewcommand{\figurename}{Рисунок}%
  \renewcommand{\contentsname}{\textnormal{Содержание}}%
}

% -------- ГЕОМЕТРИЯ И ОТСТУПЫ --------

% Полуторный интервал сделаем через setspace. Подробнее о том, как сжимать
% определенные части текста (например, текст в титульнике) с другим интервалом:
% https://ru.wikibooks.org/wiki/LaTeX/Форматирование_текста#Межстрочный_интервал
\usepackage{setspace}
\onehalfspacing

% Сделаем нужные отступы. Выдержка из ГОСТ 7.32-2017:
% Текст отчета следует печатать, соблюдая следующие размеры полей:
% левое - 30 мм,
% правое - 15 мм,
% верхнее и нижнее - 20 мм.
\usepackage[left=3cm,right=1.5cm,top=2cm,bottom=2cm]{geometry}
% Абзацный отступ должен быть
% одинаковым по всему тексту отчета и равен 1,25 см.
\setlength{\parindent}{1.25cm}

% Здесь человек предлагает использовать indentfirst для авто-отступа
% в параграфе: http://mydebianblog.blogspot.com/2006/11/latex-usepackage.html
\usepackage{indentfirst}

% Перечисления. За примером смотрим пункт 6.4.6.
\usepackage{enumitem}
\usepackage{calc}% используется для сложения длин
\setlist[itemize]{%
    leftmargin=0pt, % согласно ГОСТ, вторая строка элемента списка должна
% начинаться без абзацного отступа.
%
% Отступ первой строки от левого края будет равен: абз. отступ + ' -- '.
% Хочу заметить, что это результат не совсем точный, но выглядит неплохо.
% (другого способа настроить нужный отступ не нашел. ¯\_(ツ)_/¯). TODO.
    itemindent={\the\parindent + \widthof{\ --\ }},
    itemsep=0cm, % лол, не знаю что это;
    nolistsep, % убираем большие скачки по вертикали;
    label=--% используем короткое тире вместо bullets.
}
\setlist[enumerate]{%
% Итоговый отступ элемента от левого края будет: 1.5cm + ширина ' 1) '.
% В результате на элементе из двух цифр, типа '10)', может вылезти за края 😨.
    leftmargin=0pt,
    itemindent={\the\parindent + \widthof{\ 1)\ }},
    itemsep=0cm,
    nolistsep,
    label={\arabic*)}%
}

% -------- ФОРМАТИРОВАНИЕ ЗАГОЛОВКОВ РАЗДЕЛОВ --------

% Пункт 6.2.4: Переносы слов в заголовках не допускаются. Отключаем их.
\usepackage[raggedright,explicit]{titlesec}
% Форматируем заголовки элементов. См. подробности в документации titlesec
\titleformat{\chapter}[block]
{\bfseries}% жирный шрифт
{\hspace{1.25cm}\thechapter}% label
{\widthof{\ \ }}% два пробела между номером и названием раздела
{#1}% before-code

% Специальный случай для структурных элементов См. 6.2.1
\titleformat{name=\chapter,numberless}[block]
{\bfseries\centering}%
{}%
{0pt}%
{\MakeUppercase{#1}}

\titleformat{\section}[block]
{\bfseries}% жирный шрифт
{\hspace{1.25cm}\thesection}% label
{\widthof{\ \ }}% два пробела между номером и названием подраздела
{#1}% before-code

% На всякий пожарный, numberless-версия
\titleformat{name=\section,numberless}[block]
{\bfseries}% жирный шрифт
{\hspace{1.25cm}}% label
{\widthof{\ \ }}% два пробела между номером и названием подраздела
{#1}% before-code

% С пунктами сложнее всего. Согласно гост, они ведь и заголовки иметь могут...
% Не рекомендую, в общем, их использовать.
\titleformat{\subsection}[runin]
{\normalfont}% обычный шрифт
{\hspace{1.25cm}\thesubsection}% label
{\widthof{\ \ }}% два пробела между номером и названием подраздела
{#1}
% {\ifblank{#1}{}{%
%     \addcontentsline{toc}{subsection}{#1}
%     #1 \\ \indent%
% }}% before-code

% Теперь нормально расположим элементы по горизонтали
% \titlespacing{command}{left}{before-sep}{after-sep}
\titlespacing{\chapter}{0pt}{-20pt}{18pt}
\titlespacing{\section}{0pt}{18pt}{0pt}
\titlespacing{\subsection}{0pt}{0pt}{\wordsep}

% -------- ФОРМАТИРОВАНИЕ СОДЕРЖАНИЯ --------

% Отступы
\makeatletter
\renewcommand{\l@chapter}{\@dottedtocline{1}{2ex}{2ex}}
\renewcommand{\l@section}{\@dottedtocline{1}{4ex}{4ex}}
% \renewcommand{\l@subsection}{\@dottedtocline{1}{6ex}{6ex}}
\makeatother

% Не отображаем пункты. Что уж тут поделать.
\setcounter{tocdepth}{1}

% -------- ИЛЛЮСТРАЦИИ И ТАБЛИЦЫ --------

% предоставляет нам возможность вставлять картинки через \includegraphics
% https://www.overleaf.com/learn/latex/Inserting_Images
\usepackage{graphicx}
% Объявляем поддерживаемые форматы
\DeclareGraphicsExtensions{.pdf,.png,.jpg}

% Настройка подписей: убираем пустую строку, ставим дефис как разделитель
\usepackage[labelsep=endash]{caption}
% Смещаем подпись к таблице влево
\captionsetup[table]{singlelinecheck=false,justification=raggedright}
% Убираем пустую строку после подписей к рисункам
% \captionsetup[figure]{belowskip=-14pt,aboveskip=0pt}

% Используем расширенные таблицы (tabularx + longtable в одном).
% Вот здесь замечательный ответ, решающий большинство проблем по ним.
% https://tex.stackexchange.com/questions/59309/ltablex-customize-caption
\usepackage{ltablex}

% -------- БИБЛИОГРАФИЯ --------

% \usepackage{csquotes}
% Подключаем Библатех
\usepackage[
    backend=biber,
    sorting=none,
    style=gost-numeric
]{biblatex}

% Переопределяем заголовок списка источников
\DefineBibliographyStrings{russian}{%
  bibliography=\textnormal{Список использованных источников},
}

% Сбрасываем особенный стиль ссылок
\usepackage{url}
\urlstyle{same}

% -------- МИКРО-УЛУЧШЕНИЯ --------

% Избегает "линий-одиночек" (но порой создает пустые строки в конце страницы).
\usepackage[all]{nowidow}
% ну, или так:
% \widowpenalty=10000
% \clubpenalty=10000

% Интересный пакет. Чтобы понять его прелесть, соберите два документа -
% с этим пакетом и без, откройте их side-by-side, и внимательно посмотрите,
% чем они отличаются.
\usepackage{microtype}

% Гиперссылки
% hidelinks скрывает бесячую (неотключаемую?) подсветку ссылок в Acrobat
% и Firefox.
\usepackage[pdfusetitle,hidelinks]{hyperref}

% Условные выражения в коде (см. \newapp)
\usepackage{etoolbox}

% Позволяет использовать нижнее подчеркивание ("_") без слэша.
\usepackage{underscore}

% Подключаем полезные макросы.
% Вставка заголовка 'Введение'
\newcommand{\intro}{%
    \chapter*{Введение}\addcontentsline{toc}{chapter}{Введение}
}

% Вставка заголовка 'Заключение'
\newcommand{\conclusion}{%
    \chapter*{Заключение}\addcontentsline{toc}{chapter}{Заключение}
}

% Вставка списка источников
\newcommand{\sources}{%
    \printbibliography[heading=bibintoc]
}

% Вставка картинки.
% Пример: \insertimage[ширина]{файл_без_расширения}{Подпись}.
% Ссылка на картинку в тексте: \ref{файл_без_расширения}
\newcommand{\insertimage}[3][width=\textwidth]{%
    \begin{figure}[!ht]%
        \centering
        \includegraphics[#1]{#2}%
        \caption{#3}\label{#2}%
    \end{figure}
}

% Вставка кода.
% Пример: \begin{codewrap}[ширина]
\newenvironment{codewrap}[1][1]{
    \begin{figure}[!ht]
        \centering
            \begin{minipage}{#1\textwidth}
}{
            \end{minipage}
    \end{figure}
}

% Вставка приложения.
% Пример: \newapp{}
\newcounter{coolappcounter}
\newcommand{\newapp}[2]{%
	\refstepcounter{coolappcounter}
    \chapter*{Приложение \thecoolappcounter}\label{#1}
    \addcontentsline{toc}{chapter}{Приложение \thecoolappcounter\ifblank{#2}{}{\ \ #2} }
    \ifblank{#2}{}{%
        \vspace*{-0.7cm}
        \bfseries\centering{#2}\par
    }
    \normalfont
}
\renewcommand{\thecoolappcounter}{\Asbuk{coolappcounter}}


% Подключаем математику.
\usepackage{amsmath,amsthm,amssymb,breqn}

% Подключаем подсветку синтаксиса.
\usepackage[outputdir=build]{minted}
% \usepackage[gray]{xcolor}
\usemintedstyle{vs}
% убирает красные рамки при использовании некорректного синтаксиса.
\makeatletter
\AtBeginEnvironment{minted}{\dontdofcolorbox}
\def\dontdofcolorbox{\renewcommand\fcolorbox[4][]{##4}}
\makeatother

% Устанавливаем путь к изображениям.
\graphicspath{{images/}}
% Подключаем список лит-ры.
\addbibresource{refs.bib}

% Конец преамбулы. В окружении `document` помещается сам текст работы.
\begin{document}

% Вставим титульник из файла `title.tex`.
\begin{titlepage}
\newcommand{\ctitle}[1]{#1\title{#1}}
\newcommand{\cauthor}[1]{#1\author{#1}}
\newgeometry{left=3cm,right=0.9cm,top=2cm,bottom=2cm}
\singlespacing

\begin{center}
	\small
	Министерство науки и высшего образования Российской Федерации\\
    Федеральное государственное бюджетное образовательное учреждение\\
    высшего образования\\
\end{center}

\begin{center}
	\bfseries
	<<Пермский национальный исследовательский\\
	политехнический университет>>
\end{center}

\begin{center}
    Электротехнический факультет\\
    \mbox{Кафедра <<Информационные технологии и автоматизированные системы>>}\\
    \mbox{направление подготовки: 09.04.01 -- <<Информатика и %
    вычислительная техника>>}
\end{center}

\vspace{3em}

\begin{center}
	\fontsize{16pt}{16pt}\selectfont\bfseries
	КУРСОВАЯ РАБОТА\\
	по дисциплине\\
	<<Очень нужная дисциплина>>\\
	на тему\\
	<<\ctitle{Очень интересная тема}>>
\end{center}

\vspace{2em}

\begin{center}
\hfill
\begin{minipage}{0.48\textwidth}
    Выполнил студент гр. АСУ6-19-1м

    \cauthor{Хохряков Максим Сергеевич}
    \vskip 1cm
    \underline{\hspace{\textwidth}}

    \centering\footnotesize
    (подпись)
    \end{minipage}
\end{center}

\vspace{1em}

\begin{flushleft}
    \begin{minipage}{0.48\textwidth}
    Проверил:\\
    препод с кафедры ИТАС\\
    а как его зовут то?
    \vskip 1cm
    \begin{minipage}{0.48\textwidth}
        \centering\footnotesize
        \underline{\hspace{\textwidth}}
        (оценка)
    \end{minipage}
    \hfill
    \begin{minipage}{0.48\textwidth}
        \centering\footnotesize
        \underline{\hspace{\textwidth}}
        (подпись)
        \end{minipage}
        \vskip 1cm
        \hfill
        \begin{minipage}{0.48\textwidth}
            \centering\footnotesize
            \underline{\hspace{\textwidth}}
            (дата)
        \end{minipage}
    \end{minipage}
\end{flushleft}

\vspace{\fill}

\centering{Пермь \the\year}
\end{titlepage}
% Титульник не идет в счет страниц, исправим это вручную:
\setcounter{page}{2}

% Вставляем автоматически генерируемое содержание.
\tableofcontents

\intro

Начнем на документ с формулы \eqref{eq:wtf} в дань уважения истории,
поскольку \emph{LaTeX} создавался математиками для математиков. Другие формулы
есть в приложении \ref{app:coolformulas}.

\begin{equation}\label{eq:wtf}
 P(A) = \sum P(\{ (e_1,\dotsc,e_N) \}) = \begin{pmatrix} N\\k \end{pmatrix} \cdot p^kq^{N-k}
\end{equation}

Также можно добавить картинку, как, собственно, это показано на рисунке
\ref{randomPicture}. Заодно сразу вставим ссылку на литературу \cite{frey2004using}.
Как видно, для ссылки лит-ру используется тег \emph{\textbackslash cite}, для всего
остального -- \emph{\textbackslash ref}.

\insertimage{randomPicture}{%
    Здесь мы видим способность LaTeX вставлять картинки
}

Больше изображений можно найти в приложении \ref{app:weebs}.

Отступ после картинки должен тоже быть нормальным. Кроме того, у нас есть
рабочие длинные таблицы. Чтобы на них посмотреть, взгляните, например, на
таблицу \ref{tab:mytab}.

\noindent
\begin{tabularx}{\textwidth} {%
% Размечаем два столбца,
    |>{\small\hsize=0.35\hsize\raggedright\arraybackslash}X
    |>{\small\hsize=0.65\hsize\raggedright\arraybackslash}X|}

% вставляем название,
\caption{Этапы обработки предложения}\label{tab:mytab}\\

% указываем, как должна выглядеть шапка таблицы,
\hline
\textbf{Первый столбец} & \textbf{Второй столбец подлиннее}\\
\endfirsthead

% указываем другую шапку, которая будет появляться на следующей странице таблицы.
\caption*{Продолжение таблицы \ref{tab:mytab}}\\
\endhead

% Теперь дело только за содержимым. Не забывайте ставить двойной слэш в конце
% строки таблицы.
\hline
Оригинальное предложение & Показанный на рисунке
гидродинамический удар, безусловно, подрывает смысл жизни.\\
\hline
Токенизация & {[}'Показанный', 'на',
'рисунке', 'гидродинамический', 'удар', ',', 'безусловно', ',', 'подрывает',
'смысл', 'жизни', '.'{]} \\
\hline
Приведение слов в нормальную форму и определение частей речи & {[}'Показать',
'на', 'рисунок', 'гидродинамический', 'удар', ',', 'безусловно', ',',
'подрывать', 'смысл', 'жизнь', '.'{]} \\
\hline
\multicolumn{2}{|>{\small\raggedright\arraybackslash}X|}%
{Сплошной столбец, почему бы и нет?} \\
\hline
Выделение термов & {[}'рисунок', 'гидродинамический удар', 'смысл жизни'{]} \\
\hline
\end{tabularx}

Получилась довольно длинная таблица. Хорошо, что она закончилась, как и это
введение.

\chapter{Начало всех начал}
Что бы рассказать в этой главе? Пока не знаю, но информация будет действительно
интересной.

\section{Разметка текста}
\subsection{}
Попробуем написать что-нибудь \emph{курсивом}. В \emph{LaTeX} вроде бы
приходится экранировать \emph{нижнее_подчеркивание}, но здесь все работает и без
него благодаря пакету underscore. Жирный текст также \textbf{можно написать},
или вообще, скомбинировать \textbf{\emph{все вместе}}.
\subsection{}
Лорем ипсум -- псевдо-латинский текст, который
используется для веб дизайна, типографии, оборудования, и распечатки вместо
английского текста для того, чтобы сделать ударение не на содержание, а на
элементы дизайна.
\subsection{}
Такой текст также называется как заполнитель. Это очень
удобный инструмент для моделей (макетов). Он помогает выделить визуальные
элементы в документе или презентации, например текст, шрифт или разметка.

\section{По стандарту, переносы в
заголовках запрещены. Сам же текст вполне может быть записан в две и более строки}
Допустим, у нас есть список:
\begin{itemize}
    \item здесь расположен первый элемент;
    \item второй элемент будет здесь, при этом он может быть достаточно большим,
и латех должен справиться;
    \item третий элемент располагается вот тут;
\end{itemize}

А теперь то же самое, только с нумерованными:
\begin{enumerate}
    \item здесь расположен первый элемент;
    \item второй элемент будет здесь, при этом он может быть достаточно большим,
и латех должен справиться;
    \item третий элемент располагается вот тут;
\end{enumerate}

\section{Вставка кода}

Куда же нам, программистскому отродью без кусков кода. В ГОСТ не оговаривается,
как они должны выглядеть -- следовательно, вставляют их как картинки. Однако, не
торопитесь открывать \emph{mspaint} и сжимать джипеги -- давайте использовать
средства \emph{LaTeX}. Результат вы видите на рисунках \ref{src:quake} и \ref{src:lsof}.

\begin{codewrap}
\begin{minted}[fontsize=\footnotesize]{cpp}
// Код для расчета быстрого квадратного корня.
float Q_rsqrt( float number )
{
    long i;
    float x2, y;
    const float threehalfs = 1.5F;

    x2 = number * 0.5F;
    y  = number;
    i  = * ( long * ) &y;         // evil floating point bit level hack
    i  = 0x5f3759df - ( i >> 1 ); // what the fuck?
    y  = * ( float * ) &i;
    y  = y * ( threehalfs - ( x2 * y * y ) );

    return y;
}
\end{minted}
\caption{}\label{src:quake}
\end{codewrap}

\begin{codewrap}
\begin{minted}[fontsize=\tiny]{bash}
khokm@arch ~/Downloads % sudo lsof -i   
COMMAND     PID            USER   FD   TYPE DEVICE SIZE/OFF NODE NAME
systemd-n   217 systemd-network   20u  IPv6  16725      0t0  UDP arch:dhcpv6-client 
systemd-r   322 systemd-resolve   11u  IPv4  17459      0t0  UDP *:llmnr 
systemd-r   322 systemd-resolve   12u  IPv4  17460      0t0  TCP *:llmnr (LISTEN)
systemd-r   322 systemd-resolve   13u  IPv6  17462      0t0  UDP *:llmnr 
systemd-r   322 systemd-resolve   14u  IPv6  17463      0t0  TCP *:llmnr (LISTEN)
systemd-r   322 systemd-resolve   17u  IPv4  17465      0t0  UDP localhost:domain 
systemd-r   322 systemd-resolve   18u  IPv4  17466      0t0  TCP localhost:domain (LISTEN)
firefox   38272           khokm   99u  IPv4 553848      0t0  TCP arch:43252->93.186.225.201:https (ESTABLISHED)
firefox   38272           khokm  164u  IPv4 962950      0t0  TCP arch:41070->srv72-190-240-87.vk.com:https (ESTABLISHED)
firefox   38272           khokm  170u  IPv4 960967      0t0  TCP arch:59108->hem08s07-in-f14.1e100.net:https (ESTABLISHED)
firefox   38272           khokm  191u  IPv4 535744      0t0  TCP arch:55228->stackoverflow.com:https (ESTABLISHED)
firefox   38272           khokm  206u  IPv4 954987      0t0  TCP arch:59066->hem08s07-in-f14.1e100.net:https (ESTABLISHED)
firefox   38272           khokm  215u  IPv4 947198      0t0  TCP arch:59388->stackoverflow.com:https (ESTABLISHED)
firefox   38272           khokm  223u  IPv4 960849      0t0  TCP arch:51742->hem08s07-in-f4.1e100.net:https (ESTABLISHED)
firefox   38272           khokm  229u  IPv4 960289      0t0  TCP arch:60234->bud02s22-in-f14.1e100.net:https (ESTABLISHED)
firefox   38272           khokm  246u  IPv4 960924      0t0  TCP arch:37586->bud02s21-in-f163.1e100.net:https (ESTABLISHED)
firefox   38272           khokm  258u  IPv4 554876      0t0  TCP arch:32898->srv186-129-240-87.vk.com:https (ESTABLISHED)
firefox   38272           khokm  263u  IPv4 960938      0t0  TCP arch:54104->lm-in-f196.1e100.net:https (ESTABLISHED)
firefox   38272           khokm  290u  IPv4 960316      0t0  TCP arch:37012->mad06s10-in-f162.1e100.net:https (ESTABLISHED)
firefox   38272           khokm  292u  IPv4 962959      0t0  TCP arch:34956->hem08s06-in-f10.1e100.net:https (ESTABLISHED)
firefox   38272           khokm  303u  IPv4 899629      0t0  TCP arch:59082->stackoverflow.com:https (ESTABLISHED)
firefox   38272           khokm  304u  IPv4 963874      0t0  TCP arch:37036->mad06s10-in-f162.1e100.net:https (ESTABLISHED)
firefox   38272           khokm  309u  IPv4 945752      0t0  TCP arch:59348->stackoverflow.com:https (ESTABLISHED)
firefox   38272           khokm  340u  IPv4 844239      0t0  TCP arch:58442->stackoverflow.com:https (ESTABLISHED)
code      67227           khokm   95u  IPv4 952138      0t0  TCP localhost:32919 (LISTEN)      
\end{minted}
\caption{}\label{src:lsof}
\end{codewrap}

Результат вставки исходника из файла показан на рисунке \ref{src:flask}.

\begin{codewrap}
    \inputminted[fontsize=\footnotesize]{python}{source.py}
    \caption{Вставка куска кода}\label{src:flask}
\end{codewrap}

Все поддерживаемые синтаксисы можно найти здесь -
\url{http://pygments.org/docs/lexers/}. Кстати, вставлять реальные ссылки прямо
в текст не стоит -- добавляйте ее как ссылку на интернет-ресурс в список
источников. Все источники обязательно нужно цитировать в тексте
\cite{бобаренко2018отсечение}.

\conclusion
Lorem ipsum по большей части является элементом латинского текста классического
автора и философа Цицерона. Слова и буквы были заменены добавлением или
сокращением элементов, поэтому будет совсем неразумно пытаться передать
содержание; это не гениально, не правильно, используется даже не понятный
латинский. Хотя Lorem ipsum напоминает классический латинский, вы не найдете
никакого смысла в сказанном. Поскольку текст Цицерона не содержит буквы K, W,
или Z, что чуждо для латинского, эти буквы, а также многие другие часто
вставлены в случайном порядке, чтобы скопировать тексты различных Европейских
языков, поскольку диграфы не встречаются в оригинальных текстах

Душа моя озарена неземной радостью, как эти чудесные весенние утра, которыми я
наслаждаюсь от всего сердца. Я совсем один и блаженствую в здешнем краю, словно
созданном для таких, как я. Я так счастлив, мой друг, так упоен ощущением покоя,
что искусство мое страдает от этого. Ни одного штриха не мог бы я сделать, а
никогда не был таким большим художником, как в эти минуты. Когда от милой моей
долины поднимается пар и полдневное солнце стоит над непроницаемой чащей темного
леса и лишь редкий луч проскальзывает в его святая святых, а я лежу в высокой
траве у быстрого ручья и, прильнув к земле, вижу тысячи всевозможных былинок и
чувствую, как близок моему сердцу крошечный мирок, что снует между стебельками,
наблюдаю эти неисчислимые, непостижимые разновидности червяков и мошек и
чувствую близость всемогущего, создавшего нас по своему подобию, веяние
вселюбящего, судившего нам парить в вечном блаженстве, когда взор мой туманится
и все вокруг меня и небо надо мной запечатлены в моей душе, точно образ
возлюбленной, -- тогда, дорогой друг, меня часто томит мысль: "Ах!".

% Вставляем список источников, сгенерированный из файла refs.bib
\sources

\newapp{app:coolformulas}{Очень большое и интересное приложение, не так ли?}
Действительно...

\newapp{app:weebs}{}
А здесь название отсутствует. И обещанные картинки тоже. Печаль-бида :(.

\end{document}
